\documentclass[openany]{book}
\usepackage{amsmath}
\usepackage{amssymb}
\usepackage{amsthm}
\usepackage{enumerate}
\usepackage{bbm}
\usepackage{algorithm}
\usepackage{algorithmic}

\newtheorem{defn}{Definition}[chapter]
\newtheorem{thm}{Theorem}[chapter]
\newtheorem{lem}{Lemma}[chapter]
\newtheorem{cor}{Corollary}[chapter]
\newtheorem{exa}{Example}[chapter]
% When giving labels for above environments or algorithm environments, use the initials.

\author{Ziwei Ji}
\title{Notes on Probability and Computing}

\begin{document}
\newcommand{\tbf}[1]{\textbf{#1}}
\newcommand{\bs}[1]{\boldsymbol{#1}}
\newcommand{\mr}[1]{\mathrm{#1}}
\newcommand{\mc}[1]{\mathcal{#1}}
\newcommand{\mbb}[1]{\mathbb{#1}}
\newcommand{\mbf}[1]{\mathbf{#1}}

\maketitle

\setcounter{chapter}{12}

\chapter{Pairwise Independence and Universal Hash Functions}
\section{Pairwise Independence}
Notions of $k$-wise independent and specifically pairwise independent could be given. What's more, one can construct $2^n-1$ pairwise independent uniform binary bits from $n$ mutually independent uniform binary bits. The idea is to take the $2^n-1$ non-empty subsets of $[n]$ and perform the exclusive-or operation.

Pairwise independent random bits can be used to de-randomize a random algorithm for large cut. We assign each node independently to one side of the cut, and as a result, the expected cut size is $m/2$. Then we can use $\lceil\log_2{n+1}\rceil$ mutually independent uniform binary random bits to generate $n$ pairwise independent binary bits and assign them to the nodes. One may try all $O(n)$ possible assignments and find the cut whose size is at least $n/2$. The time complexity is $O(nm)$.


\end{document}